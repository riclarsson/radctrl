
\chapter{The workspace}
%-------------------------------
\label{sec:workspace}

\starthistory
  110622 & Updated by Oliver Lemke.\\
  020605 & Created by Stefan Buehler.\\
\stophistory

This chapter deals with the main components of ARTS: \emph{Workspace
  variables}\index{workspace variables} (\textindex{WSVs}) and
\emph{workspace methods}\index{workspace methods} (\textindex{WSMs}).
Furthermore, it explains the use of \textindex{agendas}, a special
group of WSVs.

%To make the text easier to read, we will often use \emph{variables}
%as a synonym for workspace variables, and \emph{methods} as a synonmy
%for workspace methods. So:

%\begin{quote}
%\begin{tabular}{lclcl}
%  workspace variable &=& WSV &=& variable,\\
%  workspace method   &=& WSM &=& method
%\end{tabular}
%\end{quote}

\section{Implementation files}
%----------------------------
\label{sec:agendas:files}

The most important files are:
\begin{itemize}
\item \fileindex{workspace.cc}:\\
  Definition and documentation of WSVs.
\item\fileindex{methods.cc}:\\
  Definition and documentation of WSMs. The
  implementations of WSMs reside in files named
  \shortcode{m\_something.cc}.
\item \fileindex{agendas.cc}:\\
  Definition and documentation of agendas.
\end{itemize}
It is very likely that you will have to edit these. Less likely, but
possibly, you also have to edit:
\begin{itemize}
\item \fileindex{groups.cc}:\\
  Definition of WSV groups.
\end{itemize}

\vspace{2ex}
When ARTS is built, a number of source code files are generated
automatically. They are listed here in the order in which they are
generated: 
\begin{itemize}
\item \fileindex{auto\_workspace.h}:\\
  Generated from \shortcode{groups.cc}.
\item \fileindex{auto\_md.h}, \fileindex{auto\_md.cc}:\\
  Generated from \shortcode{auto\_workspace.h},
  \shortcode{agendas.cc}, \shortcode{groups.cc}, and \shortcode{methods.cc}.
\end{itemize}
This is achieved by a set of simple C++ programs:
\begin{itemize}
\item \fileindex{make\_auto\_workspace\_h.cc}
\item \fileindex{make\_auto\_md\_h.cc}
\item \fileindex{make\_auto\_md\_cc.cc}
\end{itemize}
The meaning of the names should be self-explanatory. There is one program
for each file to be generated.  The generation of the
\shortcode{auto\_} files happens automatically when you do a
\shortcode{make}. Therefore, never edit any of these files.

Next, there are some files that contain the internal implementation
of WSVs and WSMs. These are:
\begin{itemize}
\item \fileindex{wsv\_aux.h}:\\
  Implementation of class
  \typeindex{WsvRecord}, which stores the lookup information for one
  WSV, plus auxiliary stuff for the workspace.
\item \fileindex{methods.h}, \fileindex{methods\_aux.cc}:\\
  Implementation of class \typeindex{MdRecord}, which stores the
  lookup information for one WSM.
\end{itemize}

Finally, there are some files that contain the internal implementation
of agendas. These are:
\begin{itemize}
\item \fileindex{agenda\_class.h}, \fileindex{agenda\_class.cc}:\\
  Implementation of class \typeindex{MRecord}, which stores runtime
  information for one WSM, and class \typeindex{Agenda}, which stores
  an agenda.
\item \fileindex{agenda\_record.h}, \fileindex{agenda\_record.cc}:\\
  Implementation of class \typeindex{AgRecord}, which is used to store
  agenda lookup information.
\end{itemize}
  

\vspace{2ex} As mentioned above, you will not have to modify any of
the implementation files, they are listed here just for reference.
Normally, you only have to modify \shortcode{workspace.cc},
\shortcode{methods.cc}, and \shortcode{agendas.cc}.

\section{Workspace Variables or WSVs}
%--------------------------
\label{sec:agendas:wsvs}

All important variables in ARTS are WSVs. This means that they can be
manipulated by a list of WSMs, which is specified in the ARTS
controlfile. There exists a predefined list of possible WSVs. This
list defines the \emph{workspace}. One can think of each WSV as a
`slot' in the workspace: The WSV can be either \emph{set}, or
\emph{unset}. Set means that the WSV has a well-defined content, unset
means that it has no well-defined content. At the start of an ARTS job
all WSVs are unset.

WSVs are defined in the file \fileindex{workspace.cc}. A typical
definition looks like this:

\begin{code}
wsv_data.push_back
  (WsvRecord
   ( NAME( "f_grid" ),
     DESCRIPTION
     (
      "The frequency grid for monochromatic pencil beam\n"
      "calculations.\n"
      "\n"
      "Usage:      Set by the user.\n"
      "\n"
      "Unit:       Hz"
      ),
     GROUP( "Vector" )));
\end{code}

\noindent
All WSV definitions have the same three elements:
\begin{enumerate}
\item The \emph{name}, exactly the
  same name has to be used in the code.
\item The \emph{description}, which is normally much longer than in
  the example here. It must fully describe the WSV, its purpose, and
  its normal usage. See file \shortcode{workspace.cc} for instructions
  how to write the documentation.
\item The \emph{group} to which the WSV belongs. You can think of a
  group as something similar to a C++ data type. The WSV in the
  example belongs to the group \builtindoc{Vector}. The allowed groups
  are defined in file \fileindex{groups.cc}.
\end{enumerate}

\noindent
See Section \ref{sec:development:extending} for explicit
instructions how to add a new WSV to ARTS.

\section{Workspace Methods or WSMs}
%---------------------------------
\label{sec:agendas:wsms}

WSMs manipulate WSVs to produce other WSVs. There are three kinds of
WSMs:
\begin{enumerate}
\item Specific WSMs.
\item Generic WSMs.
\item Agenda WSMs.
\end{enumerate}
As in the case of WSVs, there is a central place in ARTS where
information on the available WSMs is stored. This place is the file
\fileindex{methods.cc}. It contains a record for each WSM. Here is an
example:

\begin{code}
md_data_raw.push_back
  ( MdRecord
    ( NAME( "r_geoidSpherical" ),
      DESCRIPTION
      (
       "Sets the geoid to be a perfect sphere.\n"
       "\n"
       "The radius of the sphere is selected by the generic argument r.\n"
      ),
      AUTHORS( "Patrick Eriksson" ),
      OUT( "r_geoid" ),
      GOUT(),
      GOUT_TYPE(),
      GOUT_DESC(),
      IN( "atmosphere_dim", "lat_grid", "lon_grid" ),
      GIN( "r" ),
      GIN_TYPE(    "Numeric" ),
      GIN_DEFAULT( NODEF     ),
      GIN_DESC( "Radius of the geoid sphere."),
      ));
\end{code}

\noindent
All WSM definitions have the same elements:
\begin{enumerate}
\item The \emph{NAME}, exactly as in the code.
\item The \emph{DESCRIPTION}. This must fully describe the WSM, its
  purpose, and its normal usage. See file \shortcode{methods.cc} for
  instructions how to write the documentation.
\item The \emph{OUT}. This is a list of WSV names.
All these WSVs are set by this WSM.
\item The \emph{GOUT}. This is a list descriptive names for the generic
outputs.
\item The \emph{GOUT\_TYPE}. This is a list of WSV group names.
This defines the group to which the generic output arguments must belong
(see below).
\item The \emph{GOUT\_DESC}, a list of short descriptions for the generic outputs.
\item The \emph{IN}. This is a list of WSV names.
All these WSVs are required as input by this WSM. This means they must have
been set before.
\item The \emph{GIN}, a list of descriptive names for the generic inputs.
\item The \emph{GIN\_TYPE}. This is a list of WSV group names.
This defines the group to which the generic input arguments must belong.
\item The \emph{GIN\_DEFAULT}, a list of default values for the generic inputs.
\shortcode{NODEF} means that the generic input has no default and the user has
to set it in the control file.
\item The \emph{GIN\_DESC}, a list of short descriptions for the generic inputs.
\end{enumerate}

\subsection{Specific WSMs}
%---------------------

\begin{code}
md_data_raw.push_back     
  ( MdRecord
    ( NAME( "p_gridFromGasAbsLookup" ),
      DESCRIPTION
      (
       "Sets *p_grid* to the frequency grid of *abs_lookup*.\n"
      ),
      AUTHORS( "Patrick Eriksson" ),
      OUT( "p_grid" ),
      GOUT(),
      GOUT_TYPE(),
      GOUT_DESC(),
      IN( "abs_lookup" ),
      GIN(),
      GIN_TYPE(),
      GIN_DEFAULT(),
      GIN_DESC()
      ));
\end{code}

For this type of WSM the output and input is fixed. Fields
\shortcode{GIN} and \shortcode{GOUT} are empty. The example
above belongs in this category. It sets the WSV \builtindoc{p\_grid}, using
the WSV \builtindoc{abs\_lookup} as input.

To call this method in the controlfile, you just have to write
\builtindoc{p\_gridFromGasAbsLookup}.

\subsection{Generic WSMs}
%--------------------

This class of WSMs is more powerful, because it can be applied to any
WSV that belongs to the right group. A good example is:

\begin{code}
md_data_raw.push_back
  ( MdRecord
    ( NAME( "VectorSetConstant" ),
      DESCRIPTION
      (
       "Creates a vector and sets all elements to the specified value.\n"
       "\n"
       "The vector length is determined by *nelem*.\n"
       ),
      AUTHORS( "Patrick Eriksson" ),
      OUT(),
      GOUT(      "v"      ),
      GOUT_TYPE( "Vector" ),
      GOUT_DESC( "Variable to initialize." ),
      IN( "nelem" ),
      GIN(         "value"   ),
      GIN_TYPE(    "Numeric" ),
      GIN_DEFAULT( NODEF     ),
      GIN_DESC(    "Vector value." )
      ));
\end{code}

\noindent
As you probably have guessed, this WSM resizes the output vector to have \builtindoc{nelem} elements and sets all elements to the given \shortcode{value}.
You would use it as follows:

\begin{code}
IndexSet(nelem, 10)
VectorCreate(myvector)
VectorSetConstant(myvector, nelem, 0)
\end{code}

\noindent
This would create the WSV \shortcode{myvector} and then fill it with
10 elements set to 1. Note that output arguments always come first,
input arguments last. Try \shortcode{arts -d VectorSetConstant} to get
more information on this method. (See Section
\ref{U-sec:built-in_doc} in \user\ for information on the built-in
documentation.)

\noindent
For basic types it is allowed to pass values instead of variables directly to
the WSM. In that case, the above example would look like this:

\begin{code}
VectorCreate(myvector)
VectorSetConstant(myvector, 10, 0)
\end{code}

\subsection{Agenda WSMs}
%---------------------------------

\section{Agendas}
%--------------------------
\label{sec:agendas:agendas}

\subsection{Introduction}
Agendas are a special incarnation of a WSM. At runtime an arbitrary
number of WSMs can be added to an agenda. On invocation, the agenda
will execute its methods one after the other. The inputs and outputs
defined for the agenda must be satisfied by the invoked WSMs. E.g., if
an agenda has \builtindoc{f\_grid} in its list of output WSVs, a WSM
which generates \builtindoc{f\_grid} must be added to the agenda in the
control file.

Agendas run their methods in a separate scope. Although WSMs invoked by
an agenda have full access to all workspace variables, only the WSVs
defined as output of the agenda will keep their values after the agenda
execution. All other WSVs retain the values from before the agenda
run.

Even though it is possible to execute agendas directly from the control
file with the \builtindoc{AgendaExecute} method, the more common and
intended use case is the internal invocation by other WSMs. This adds a
considerable amount of flexibility to arts.
The \builtindoc{iyEmissionStandard} method for example calculates
(besides other components) the emission term. Without the means of an agenda,
it would only be possible to use always the same method for the emission
calculation. By the use of an agenda the user can choose between different
methods to calculate the emission and plug them into the emission agenda in
the control file:

\begin{code}
AgendaSet( blackbody\_radiation\_agenda ){
  blackbody\_radiationPlanck
}
\end{code}


%\section{ARTS online documentation}
%%---------------------------------
%\label{sec:online-docu}


%%% Local Variables: 
%%% mode: latex
%%% TeX-master: "uguide"
%%% End: 

