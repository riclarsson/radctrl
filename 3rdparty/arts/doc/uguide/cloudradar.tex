\chapter{Active measurements}
 \label{sec:cradar}


\starthistory
 170609 & Started updates following recent changes.\\
 121108 & Written by Patrick Eriksson.\\
\stophistory

ARTS provides some support for modelling measurements of backscattering inside
the atmosphere. Such observations are for simplicity denoted as ``active''.
These radar or lidar measurements deviate for the transmission observations
discussed in the previous chapter, as in this case the transmitter and receiver
are placed at the same position. Here backscattering is recorded, while in the
transmission case the attenuation through the atmosphere is probed. This has
the consequence that the measurement data not only span the frequency and a
polarisation dimensions, but also has a time (or distance) dimension. That is,
the backscattering is basically reported as a function of distance from the
sensor, for one or several frequencies frequencies and combinations of
transmitted and received polarisations. Accordingly, these data differ in
nature from the other types of measurements, and the standard main function
(\builtindoc{yCalc}) is not applicable.



\section{Single scattering}
%===================
\label{sec:cradar:single}

The fastest method for this observation approach is
\wsmindex{iyActiveSingleScat}. As the name of the method indicates a very
important restriction applies, only single scattering is treated (but the
attenuation due to particles is also considered). This is frequently an
acceptable simplification for precipitation and cloud radars observations, and
such observations should be the main applications of this workspace methods.
The basic measurement approach is the same for lidars, but the assumption of
single scattering is much more restrictive for such instruments.

\subsection{Theory}
%===================
\label{sec:cradar:theory}

The transmitted pulses are treated to be monochromatic pencil beams. Effects
due to antenna patterns and the geometrical distance between the instrument and
the scattering particles are assumed to be treated as a separate
``calibration'', and the ``forward model'' treats only the actual
backscattering and atmospheric extinction. 

For the conditions given above, the backscattered radiation, \aStoVec{b}, can be
written as
\begin{equation}
  \label{eq:cradar:bscatt1}
  \aStoVec{b} = \aTraMat{h}\PhaMat^b\aTraMat{a}\aStoVec{t}.
\end{equation}
The terms in this equation are:
\begin{itemize}
\item[\aStoVec{t}] Stokes vector describing the transmitted pulse.
\item[\aTraMat{a}] A matrix describing the transmission from the receiver to
  the scattering point (the away direction). 
\item[\PhaMat$^b$] The bulk scattering (or phase) matrix value
  for the backward direction.
\item[\aTraMat{h}] As \aTraMat{a}, but for the reversed direction (the home
  direction). Note that for vector radiative transfer, in general
  $\aTraMat{a}\neq\aTraMat{h}$.
\end{itemize}
The (attenuated) ``backscatter coefficient'' recorded by the receiver is
\begin{equation}
  \label{eq:cradar:bscatt2}
  \beta' = \VctStl{p} \cdot \aStoVec{b} 
\end{equation}
where \VctStl{p} is the (normalised) vector describing the polarisation
response of the receiver and $\cdot$ signifies the dot product.

The corresponding unattenuated backscattering coefficient is
\begin{equation}
  \label{eq:cradar:bscatt3}
  \beta = \VctStl{p} \cdot \left[\PhaMat(\Omega=0)\aStoVec{t}\right]
\end{equation}


\subsection{Units}
%===================
\label{sec:cradar:units}

The unit of $\beta$ (and $\beta'$) is 1/(m sr$^{-1}$). For radar applications
this is not the most common choice, but is here preferred as it directly
matches \PhaMat. It is also the standard definition in the lidar community. The
more common definition of radar reflectivity is simply $4\pi\beta$.

However, even more common is to report radar data in unit of equivalent
reflectivity, $Z_e$. This quantity is defined as \citep[e.g.][]{donovan:01}
\begin{equation}
  \label{eq:cradar:ze}
  Z_e = \frac{4\lambda_r}{\pi^4|K|^2}\beta
\end{equation}
where $\lambda_r$ is wavelength of the radar and the ``reference dielectric
factor'' is calculated using the complex refractive index of ice or liquid
water, $n$:
\begin{equation}
  K = \frac{n^2-1}{n^2+2}.
\end{equation}



\subsection{Practical usage}
%===================
\label{sec:cradar:usage}

As for other measurements, the main radiative transfer calculations are
performed inside \builtindoc{iy\_main\_agenda}. When using
\builtindoc{iyActiveSingleScat}, \aStoVec{b} is returned
for each point of the propagation path, and for each frequency in
\builtindoc{f\_grid}. The calculated data are packed into \builtindoc{iy}. Here
the difference to other measurement types emerge. For example, the second row
holds \aStoVec{b} corresponding to the second point of the propagation path
(not the second frequency). If \builtindoc{f\_grid} contains several
frequencies, the data for the second frequency are placed below (in the row
dimension) the data for the first frequency etc. 

The polarisation of the transmitted pulses (\aStoVec{b}) are taken from
\wsaindex{iy\_transmitter\_agenda}. The agenda shall here return a Stokes
vector for each frequency in \builtindoc{f\_grid}, of unit intensity.

The unit of returned data is selected by \wsvindex{iy\_unit}. There are two
options ``1'' and ``Ze''. For the first option no unit conversion is performed,
while for the second option Equation~\ref{eq:cradar:ze} is applied on all
Stokes elements of \aStoVec{b}. In the later case, liquid water at a user
specified temperature is used as reference. That is, $K$ is calculated for the
refractive index of water at the specified temperature.

Hence, the basic calculations are performed in standard manner, using
\builtindoc{iy\_main\_agenda}. However, the deviating data pattern in
\wsvindex{iy} results in that an alternative to \builtindoc{yCalc} is needed
and it is \wsmindex{yActive}. The method applies two instrument effects.
Firstly, the polarisation response of the receiver is incorporated
(Eq.~\ref{eq:cradar:bscatt2}). Secondly, the data are averaged as a
function of ``range'', following the \wsvindex{range\_bins}. These range bins
can be specified either as geometrical altitude or the two-way propagation time.
The range binning is described further in the built-in documentation.
Further, the data are rearranged into a vector and returned as \builtindoc{y}.

A number of auxiliary data can be obtained by \builtindoc{iyActiveSingleScat}, this
including $\beta$. For a list of possible variables:
\begin{code}
arts -d iyActiveSingleScat
\end{code}
A difference of \builtindoc{yActive}, compared to \builtindoc{yCalc}, is
that all auxiliary quantities provided by \builtindoc{iyActiveSingleScat} are
treated.



\subsection{Jacobian calculations}
%===================
\label{sec:cradar:jac}

For better clarity, Eq.~\ref{eq:cradar:bscatt1} is expanded to make dependency
on particle number densities clear:
\begin{equation}
  \aStoVec{b} = \aTraMat{h}\sum_i\left(n_i<\PhaMat^b_i>\right)\aTraMat{a}\aStoVec{t}.
\end{equation}
where $n_i$ is the particle number density of scattering element $i$ and
$<\PhaMat^b_i>$ is the back-scattering cross-section of individual particles.

For an element of the state vector, \aSttElm{p}, that influences $\PhaMat^b$
(beside temperature, see below) the main expression for the Jacobian is:
\begin{equation}
  \frac{\partial\aStoVec{b}}{\partial\aSttElm{p}} = \aTraMat{h}
    \frac{\partial\PhaMat^b}{\partial\aSttElm{p}}\aTraMat{a}\aStoVec{t} = 
    \aTraMat{h} \sum_i\left(\frac{\partial n_i}{\partial\aSttElm{p}}<\PhaMat^b_i>\right)
    \aTraMat{a}\aStoVec{t}.
  \label{eq:wfuns:cradar1}
\end{equation}
If for example \aSttElm{p} represents ice water content at an altitude, the
expression above covers the effect on $\PhaMat^b$ at the same altitude. There
is no effect of \aSttElm{p} on $\PhaMat^b$ at other altitudes, and this part of
the Jacobian becomes zero. The terms $\partial n_i/\partial\aSttElm{p}$ are
taken from the workspace variable \wsvindex{dpnd\_field\_dx}.

The back-scattering has a weak temperature dependency. Some particle size
distributions have temperature as input, and Eq.~\ref{eq:wfuns:cradar1} is then
also valid for temperature. So far both these aspects are ignored, and the
temperature Jacobian is set to zero.

Changes in both absorption and scattering species can affect \aTraMat{a} and
\aTraMat{h}. The Jacobian corresponding to this effect is:
\begin{equation}
  \frac{\partial\aStoVec{b}}{\partial\aSttElm{p}} = 
   \frac{\partial\aTraMat{h}}{\partial\aSttElm{p}}\PhaMat^b\aTraMat{a}\aStoVec{t} +
   \aTraMat{h}\PhaMat^b\frac{\partial\aTraMat{a}}{\partial\aSttElm{p}}\aStoVec{t} 
   \approx 2 \frac{\partial\aTraMat{h}}{\partial\aSttElm{p}}\PhaMat^b
   \aTraMat{a}\aStoVec{t}.
  \label{eq:wfuns:cradar2}
\end{equation}
The last, approximate, expression is used, which assumes that the attenuation
not induces significant polarisation. For this condition
$\aTraMat{a}\approx\aTraMat{h}$. The derivative
$\partial\aTraMat{h}/\partial\aSttElm{p}$ is calculated following
Sec.~\ref{sec:wfuns:atmvars}





\section{Multiple scattering}
%===================
\label{sec:cradar:multi}

Radar measurements involving multiple scattering can be simulated by using
\wsmindex{MCRadar}. To be written \dots
